\subsection{Stromsparmaßnahmen}
    Da die LED Kerze über einen Akku betrieben werden soll, soll möglichst viel
    Strom gespart werden. Dabei muss abgewogen werden, was wirklich Sinnvoll ist
    und was in Relation zu anderen großen Stromverbrauchern wie der LED nur
    marginale Verbesserungen mit sich bringt.

    \subsubsection{MPU6050}
        Da von dem Sensor nur das Accelerometer verwendet wird, werden das
        Gyroskop sowie der Temperatursensor deaktiviert. Zusätzlich wird die
        Samplerate auf 32Hz runter gesetzt.

    \subsubsection{Atmega168}
        Sobald die LED nicht leuchtet, die Zustandsmaschine sich somit im
        \textit{LED\_OFF} Zustand befindet, wird der Mikrocontroller schlafen
        gelegt. Dabei wird dieser in den \textit{Standbymode} versetzt, sodass
        die Stromaufnahe von ca. $8.5mA$ auf ca. $2.5mA$ des Gesamtsystems zurück
        geht. Dies ist eine Reduktion von $70\%$ und wird die Akkulaufzeit
        verlängern.
        Dies geschieht über den Aufruf der \texttt{sleep} Funktion.

        \lstinputlisting[caption=sleep,firstline=102, lastline=112]{../main.c}
        Dabei werden zunächst alle Interrupts gesperrt, sodass es zu keinen
        \textit{Race Conditions} kommen kann. Ein Beispiel für solch eine
        \textit{Race Condition} wäre, dass die \textit{ISR} die externen
        interrupts deaktiviert, die CPU sich aber im nächsten Moment schlafen legt
        und somit nicht wieder aufwachen kann.\\
        Als nächstes wird das externe Interrupt maskiert, sodass der
        Mikrocontroller nur aufwacht, sobald am \textit{INT0} Pin eine
        steigende Flanke anliegt.
        Durch das Ausführen der \texttt{sei} Funktion werden alle Interrupts wieder aktiviert.
        Zusätzlich wird der nächste Funktionsaufruf unterbrechungsfrei garantiert.
        Somit kann es zu keinen \textit{Race Conditions} mehr kommen und der
        Mikrocontroller kann durch \texttt{sleep\_cpu} schlafen gelegt werden.

        \lstinputlisting[caption=Aufwach ISR,firstline=130, lastline=136]{../main.c}
        Der \textit{Standbymode} wurde dabei ausgewählt, da aus diesem
        der Mikrocontroller in nur sechs Taktzyklen wieder aufwachen kann.
        Das Aufwachen passiert dabei über das \textit{INT0} Interrupt welches
        durch den \textit{MPU6050} ausgelöst wird sobald diesem neue Daten zu
        Verfügung stehen.
        Nach dem Aufwachen wird der Code nach \texttt{sleep\_disbale}
        weiter ausgeführt.

