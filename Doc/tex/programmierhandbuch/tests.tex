\section{Programmtests}
    Aufgrund der niedrigen Komplexität des Programmes wurden keine automatischen
    Tests geschrieben. Die Tests wurden daher \textit{per Hand} ausgeführt.
    Dabei wurden alle Funktionen sowohl einzeln als auch im Gesamten getestet 
    und die Ergebnisse über die Debugausgabe überprüft. Zusätzlich wurde 
    die Kerze von Dritten getestet um Betriebsblindheit auszuschließen.

    \subsection{Durchgeführte Tests}
        Alle ausgeführten Tests wurden mit einer zurückgesetzten Kerze getestet,
        sodass sich die Tests gegenseiten nicht beeinflussen.

        \begin{center}
            \begin{longtable}{| p{0.4\textwidth} | p{0.3\textwidth} | p{0.3\textwidth} |}
                \hline
                Testfall & Erwartetes Ergebnis & Erzieltes Ergebnis \\
                \hline

                Starten des Laternenmodus durch Schütteln der Kerze. &
                Die Kerze sollte in der Standardfarbe (Orange), Standardhelligkeit (50\%) leuchten
                und nach der Standardabschaltverzögerung (1 min) abschalten. &
                Die Kerze leuchtet mit ihren Standartwerten und schaltet sich
                nach der Standardabschaltverzögerung ab. \\
                \hline

                Starten des Dekorationsmodus durch nach hinten Kippen der Kerze.&
                Die Kerze sollte in der Standardfarbe (Orange), Standardhelligkeit (50\%) leuchten
                und nach der Standardabschaltverzögerung (30 min) abschalten. &
                Die Kerze leuchtet mit ihren Standartwerten und schaltet sich
                nach der Standardabschaltverzögerung ab. \\
                \hline

                Zurücksetzen der Abschaltverzögerung durch Bewegung im Laternenmodus.
                Dafür wird die Kerze in Laternenmodus versetzt und nach 30 Sekunden
                geschüttelt. &
                Die Kerze sollte nach insgesamt einer Minute und 30 Sekunden nach Testbeginn
                sich ausschalten. &
                Die Kerze schaltet sich nach 1 Minute und 30 Sekunden aus.\\
                \hline

                Ausschalten der Kerze im Laternenmodus. Dafür wird die Kerze
                in den Laternenmodus versetzt und anschließend auf dem Kopf gedreht.&
                Die Kerze sollte sich ausschalten.&
                Die Kerze schaltet sich aus.\\
                \hline

                Ausschalten der Kerze im Dekorationsmodus. Dafür wird die Kerze
                in den Dekorationsmodus versetzt und anschließend auf dem Kopf gedreht.&
                Die Kerze sollte sich ausschalten.&
                Die Kerze schaltet sich aus.\\
                \hline

                Folgende Parameter werden durch den Konfigurationsmodus gesetzt.
                \begin{itemize}
                    \item Abschaltverzögerung: 3 min
                    \item Farbe: Pink
                    \item Helligkeit: 100\%
                \end{itemize}&
                Die Kerze soll in den konfigurierten Parametern leuchten und nach
                der konfigurierten Abschaltverzögerung sich abschalten.&
                Die Kerze leuchtet mit den konfigurierten Parametern und
                schaltet sich nach der konfigurierten Abschaltverzögerung ab.\\
                \hline

                Kombination der Tests. Die Tests werden ohne Zurücksetze
                hintereinander ausgeführt.&
                Die Kerze soll sich in den Tests konsistent und deterministisch verhalten&
                Die Kerze verhält sich in den Tests konsistent und deterministisch\\
                \hline

            \end{longtable}
        \end{center}