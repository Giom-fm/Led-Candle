  \subsection{Abschaltverzögerung}
    Um die Abschaltverzögerung zu realisieren wird der Hardware Timer
    \textit{Timer0} im CTC-Modus des \textit{Atmega168} benutzt.
    Dieser soll jede Millisekunde eine \textit{Interrupt Service Routine (ISR)}
    ausführen die Millisekunden, Sekunden und Stunden verwaltet.
    \lstinputlisting[caption=Abschaltverzögerung,firstline=40, lastline=56]{../timer.c}
    Um das Vergleichsregister einzustellen, sodass die
    \textit{ISR} jede Millisekunde ausgeführt wird, wird folgende Formel
    benutzt.
    $$ OCR0A = \frac{CPU_{freq}}{Vorteiler \cdot Timer_{freq}} $$
    Da die CPU Frequenz bei 1MHz liegt, für den Vorteiler acht gewählt
    wurde und die gewünschte Timer Frequenz bei 1000Hz liegt, ergibt
    sich 125 für das vergleichs Register.
    $$ 125 = \frac{1MHz}{8 \cdot 1Khz} $$
    Da jedoch keine Millisekunden genaue Auflösung benötigt wird, könnte
    auch ein anderer Wert für das Vergleichsregister und den Vorteiler
    gewählt werden, sodass die \textit{ISR} weniger oft ausgeführt wird
    und somit die CPU Auslastung geringer wird.
    Beispielsweise würde das Wählen des Vergleichsregisters von 250
    dazu führen, dass die \textit{ISR} nur noch mit einer Frequenz von
    500Hz ausgeführt wird. Jedoch müsste dann \texttt{timer\_milliseconds} um zwei
    inkrementiert werden.
