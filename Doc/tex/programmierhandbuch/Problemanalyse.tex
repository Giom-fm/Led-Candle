\section{Problemanalyse}
    \subsection{Betriebsmodi}

        \subsubsection{Laternenmodus}
            Sobald die LED Kerze bewegt wird, wird in den Laternenmodus
            geschaltet werden. Die Kerze leuchtet solange bis keine Bewegung
            mehr detektiert wurde und die Abschaltverzögerung abgelaufen ist.
            Es muss jedoch auch manuell möglich sein, die Kerze abzuschalten.

        \subsubsection{Dekorationsmodus}
            Die Kerze wird manuell in den Dekorationsmodus geschaltet.
            In diesem Modus leuchtet die LED durchgehend bis die
            Abschaltverzögerung die Kerze abschaltet. Wie im
            Laternenmodus soll es möglich sein, die Kerze manuell auszuschalten.


    \subsection{Konfiguration}
        Da die Kerze ausschließlich über Bewegungen konfigurierbar sein soll,
        wird eine Bewegungssteuerung benötigt.
        Dabei wird zwischen Bewegungen entlang einer Achse und Bewegungen
        um eine Achse unterschieden.\\
        Ein Accelerometer kann dabei nur Bewegungen entlang einer Achse,
        wie zum Beispiel das verschieben eines Körpers, messen.
        Dagegen kann ein Gyroskop nur Bewegungen um eine Achse,
        wie zum Beispiel das Rotieren eines Körpers, messen.
        Durch die Kombination der beiden Sensoren erhält man eine
        inertiale Messeinheit und kann somit jede Bewegung eines Körpers
        detektieren. Falls die Startposition sowie Startorientierung bekannt sind,
        kann damit die Position sowie die Orientierung des Körpers im 
        Raum bestimmt werden.
        Somit können Bewegungen durch einen Sensor erkannt werden und als
        Steuerung der Kerze benutzt werden.

    \subsection{Abschaltverzögerung}
        Damit es möglich ist, dass die Kerze sich automatisch nach einer
        konfigurierten Abschaltverzögerung abschaltet, ist es erforderlich,
        dass die vergangene Zeit gemessen werden kann.

    \subsection{Helligkeit und Farbe}
        Über Bewegungssteuerung soll es möglich sein, unter anderem die
        Helligkeit und Farbe einzustellen.
        Dafür wird ein puls-weiten-modulations Verfahren benötigt um die
        einzelnen Farbkanäle der RGB-LED anzusteuern.

    \subsection{Debugging}
        Es muss möglich sein das Programm zu Debuggen. Dazu wird ein Modul
        benötigt, dass Debugausgaben an einem Computer senden kann.

        




