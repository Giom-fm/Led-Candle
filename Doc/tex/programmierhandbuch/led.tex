\subsection{LED}
    Als LED kommt eine \textit{HV-5RGBXX} zum Einsatz. Diese wird über drei Pins
    mit jeweiligen 220Ohm Widerstand an den \textit{Atmega168} angeschlossen
    (Siehe \ref{schaltplan}).

\subsubsection{Farbe und Helligkeit}
    Durch die drei Pins die die Grundfarben Rot, Grün und Blau repräsentieren,
    lassen sich über additive Farbmischung unterschiedliche Farben einstellen.
    \\
    Um Intensitäten einstellen zu können wird ein puls-weiten-modulations (PWM)
    Signal pro Pin erzeugt. Hierfür wird der Hardware Timer \textit{TIMER2} des
    \textit{Atmega168} verwendet. Da der Timer jedoch nur zwei Ausgänge bestitz, kann
    kein Hardware PWM Signal, wie zum Beispiel \textit{Fast PWM} verwendet werden,
    sodass ein \textit{Soft-PWM} implementiert wurde.
    
    \lstinputlisting[caption=led\_on,firstline=49, lastline=64]{../led.c}
    Um die LED einzuschalten wird die \texttt{led\_on} Funktion mit den
    jeweiligen Parametern aufgerufen. Zunächst werden die
    Pulsweiten für die einzelnen Farbkanäle berechnet. Diese sagen aus, wie 
    lange der Farbkanal pro Periode eingeschaltet ist. In der Formel wird direkt
    die Helligkeit mit der Farbintensität verrechnet.

    \begin{displaymath}
        Pulsweite = \frac{Helligkeit \cdot Farbintensität}{Farbintensität_{max} \cdot Periodenschritte_{max}}
    \end{displaymath}
    Des weiteren wird der Timer gestartet, sodass dieser mit einer Frequenz von
    $3906Hz$ bzw. alle $256\mu{s}$ die \textit{ISR} ausführt.
    \newpage
    \lstinputlisting[caption=PWM ISR,firstline=75, lastline=115]{../led.c}
    In dieser \textit{ISR} wird der Zähler \texttt{pwm\_step} pro Aufruf inkrementiert und 
    mit den jeweiligen Pulsweiten der Farbkanäle verglichen.
    Liegt der Zähler unterhalb einer Pulsweite, wird der entsprechende Farbkanal
    eingeschaltet. Umgekehrt wird der Farbkanal ausgeschaltet sobald der Zähler
    oberhalb der Pulsweite liegt. Sobald der Zähler die maximalen Periodenschritte
    erreicht hat, wird dieser auf null zurückgesetzt.
    Als maximal Periodenschritte wurde zehn gewählt. Daraus folgt eine Periodendauer
    von ca. 2ms, welches zu schnell für das menschliche Auge ist und somit 
    als kontinuierliches Licht wahrgenommen wird.
    \\\\
    Auch wenn die \textit{TIMER2\_COMPA\_vect\ ISR} nicht rechenintensiv ist, könnte sie verkürzt
    werden, sodass die CPU weniger blockiert. Dazu könnte die Logik in die
    Hauptschleife ausgelagert werden und \texttt{OCR2A} auf die maximale
    Periodenschritte gesetzt werden. Ein vergleich des Periodenschrittes mit
    \texttt{TCNT2} wäre dann äquivalent. Der Vorteil wäre hierbei, dass die
    zeitlich kürzere \textit{ISR} die CPU weniger blockiert.
    Ein jedoch großer Nachteil könnte sein,
    dass die Farben bei hoher CPU Auslastung driften, da nicht genug CPU Zeit
    in der Hauptschleife vorhanden ist um die PWM Signale stabil zu halten.
    \\
    Ein weiterer Ansatz wäre es, zwei hardware Timer zu benutzen und
    \textit{Fast-PWM} einzusetzen. Dabei würde zum Beispiel ein Timer zwei
    Farbkanäle und der andere einen Farbkanal steuern.
    Der Vorteil an dieser Lösung wäre, dass weniger Overhead durch das aufrufen
    der \textit{ISR} auftreten würde. Jedoch werden auch zwei der drei hardware
    Timer des \textit{Atmega168} dafür beansprucht.

\subsubsection{Blinken}
    Um ein Blinken der LED zu realisieren wird der hardware Timer \textit{TIMER1}
    eingesetzt und so konfiguriert, dass seine \textit{TIMER1\_COMPA\_vect ISR}
    ca. mit 4Hz, also vier mal in einer Sekunde blinkt.
    \lstinputlisting[caption=Blink ISR,firstline=42, lastline=47]{../led.c}
    Dabei toggelt die \textit{ISR} das \texttt{led\_blink\_state} Flag und
    manipuliert somit die \\\textit{TIMER2\_COMPA\_vect\ ISR},
    sodass diese die LED komplett ausschaltet wenn das Flag nicht gesetzt ist.
    \lstinputlisting[caption=Blinkbedingung,firstline=82, lastline=85]{../led.c}
    

