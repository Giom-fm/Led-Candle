\subsection{Bewegungssteuerung}
  Zum detektieren von Bewegungen wird ein \textit{MPU6050} eingesetzt.
  Dieser Sensor beinhaltet ein drei-Achsen Gyroskop, ein drei-Achsen 
  Accelerometer und ein Temperatursensor.

    \subsubsection{Sensorwahl}
      Da die konstante Erdbeschleunigung von $1g = 9,81\frac{m}{s}$
      die Kerze kontinuierlich in Richtung Erdmittelpunkt beschleunigt,
      reicht das drei-Achsen Accelerometer aus, um die genaue
      Neigung der Kerze zu bestimmen. Zu beachten ist jedoch, dass nur
      Neigungen um die X- und Y-Achsen detektiert werden können.
      Rotationen um die Z-Achse werden von dem Accelerometer nicht
      wahrgenommen da die Z-Achse auf den beiden anderen Achsen steht.
      \\\\
      Alternativ könnte auch das Gyroskop verwendet werden. Darüber lassen
      sich Drehwinkel um die Achsen messen. Da jedoch ein Gyroskop
      $\frac{Winkel}{Zeit}$ als Einheit hat, müsste integriert und die Werte
      aufsummiert werden. Die wahrscheinlichkeit, dass der Sensorwert dabei
      driftet ist sehr hoch und wird somit nicht als Lösung betrachtet.


    \subsubsection{Achsenorentierung}\label{Achsenorentierung}
      Zu beachten ist, dass der Sensor nicht über seine Pins im
      Breakout Board befestigt wird sondern flach auf dem Board
      festgeklebt wird. Daraus ergibt sich folgende Orientierung der
      Achsen.
      \begin{description}       
          \item[Z-Achse] Senkrechte Achse
          \item[Y-Achse] Horizontale Achse
          \item[X-Achse] Senkrecht auf beiden anderen Achsen
      \end{description}



    \subsubsection{Kommunikation}
      Die Sensordaten werden über ein $TWI$ Bus an den Mikrocontroller
      übertragen.

      \paragraph{Senden}
        Als Beispiel eines Sendevorgangs wird die Konfiguration des
        Accelerometers benutzt.
        \lstinputlisting[caption=TWI Sendevorgang,firstline=26, lastline=39]{../mpu6050.c}
        Zunächst wird das Startsignal auf den Bus gelegt und gewartet bis der
        Bus frei ist. Sobald die Kondition eintritt, wird die Schreibadresse des
        \textit{MPU6050} gesetzt. Als nächstes wird die Registeradresse in die
        die Daten geschrieben werden sollen auf den Bus geschrieben und auf das
        \textit{ACK} gewartet. War dies erfolgreich, können die eigentlichen
        Daten auf den Bus geschrieben werden und das Stoppsignal gesendet werden.


      \paragraph{Empfangen}
        Als Beispiel für das Empfangen von Daten wird das auslesen des
        Accelerometers benutzt.
        \lstinputlisting[caption=TWI Empfangsvorgang,firstline=113, lastline=132]{../mpu6050.c}
        Zunächst wird wie beim Senden das Startsignal auf den Bus gelegt,
        die Schreibadresse des \textit{MPU6050} gesetzt und das zu lesenden
        Register gesendet. Nun erfolgt ein wiederholter Start mit der Leseadresse
        des \textit{MPU6050}. Durch \texttt{twi\_read\_ack} wird das erste Byte
        ausgelesen. Der \textit{MPU6050} erhöht nach dem Empfangen des \textit{ACK}
        automatisch die Registeradresse, sodass mit \texttt{twi\_read\_nack}
        das zweite Byte ausgelesen werden kann. Nach dem zweiten Byte wird
        jedoch ein \textit{NACK} gesendet, welches dem \textit{MPU6050} signalisiert,
        dass keine Daten mehr erwartet werden. Nach dem Stoppsignal ist die
        Kommunikation beendet.


    \subsubsection{Kalibrierung}
      Durch die Kalibrierung wird der \textit{null Fehler} kompensiert und damit 
      gewährleistet, dass der Sensor kalibrierte Daten liefert.
      \\
      Ein \textit{null Fehler} entsteht dadurch, dass ein Sensor einen Wert
      liefert der ungleich null ist, obwohl der Sensor sich in Ruhelage
      befindet. Da der Fehler sich selbst bei  baugleichen Sensoren
      unterscheidet, muss für jeden Sensor eine einmalige Kalibrierung erfolgen.
      \\\\
      Über die \texttt{mpu6050\_calibrate} Funktion werden die Werte gemittelt
      und dann über die \texttt{debug\_print} Funktion ausgegeben.
      Dazu wird zunächst der Sensor waagerecht auf eine plane Fläche gelegt,
      sodass die Z-Achse nach oben Zeigt und die Funktion ausgeführt.
      Nun wir der Z-Achsenwert notiert, das Verfahren für alle Achsen
      wiederholt und der entsprechende Achsenwert notiert.
      \\
      Um den eigentlichen Versatz zu berechnen werden folgende Formeln verwendet.
      $$  Versatz_X = 0 - Achsenwert_X $$
      $$  Versatz_Y = 0 - Achsenwert_Y $$
      $$  Versatz_Z = 1 - Achsenwert_Z $$
      Für den benutzen Sensor ergeben sich somit folgende Versätze.
      \begin{center}
        \begin{tabular}{| c | c |}
            \hline
            Achse & Versatz \\
            \hline
            X & $-0.035939$ \\
            \hline
            Y & $0.010786$ \\
            \hline
            Z & $0.187407$ \\
            \hline
        \end{tabular}
      \end{center}
      Der Versatz muss nun auf jeden entsprechenden Wert hinzuaddiert werden um
      den \textit{null Fehler} zu kompensieren.
      \lstinputlisting[caption=Nullfehler Kompension,firstline=136, lastline=138]{../mpu6050.c}


    \subsubsection{Digitale Tiefpassfilterung}
      Um dem Rauschen des Sensors entgegen zu wirken, wird die interne digitale
      Tiefpassfilterung des \textit{MPU6050} benutzt. Dabei wird der Tiefpassfilter
      auf $5Hz$ gestellt um hochfrequentes Rauschen zu filtern.

 



      