\part{Problemstellung}
    In dem Projekt Mikrocontroller soll mit einem Mikrocontroller 
    eine sensorgesteuerte LED-Kerze entwickelt werden die sich in zwei
    Betriebsmodi betreiben lässt.
    Dabei wird zwischen dem \textit{Laternenmodus} und dem
    \textit{Dekorationsmodus} unterschieden. Im ersteren soll die Kerze anfangen
    zu leuchten sobald sie Bewegt wird.
    Wurde keine Bewegung mehr detektiert, schaltet sich die Kerze nach Ablauf
    einer konfigurierten Abschaltverzögerung ab.
    Im zweiten Modus leuchtet die Kerze solange bis die Abschaltverzögerung
    abgelaufen ist.\\
    Dabei soll die Kerze über eine Bewegungssteuerung und nicht
    über herkömmliche Knöpfe konfiguriert werden können.
    Die Konfigurationsmöglichkeiten beinhalten Abschaltverzögerung, Helligkeit
    und Farbe. Da die Kerze über einen Akku betrieben wird, soll zusätzlich
    dafür gesorgt werden, dass sie über Stromsparmaßnahmen verfügt.

