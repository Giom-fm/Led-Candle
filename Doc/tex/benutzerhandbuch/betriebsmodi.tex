\section{Betriebsmodi}

    \subsection{Laternenmodus}
        Im Laternenmodus schaltet sich die Kerze an, sobald diese bewegt wird.
        Detektiert die Kerze keine Bewegung mehr, schaltet sie sich automatisch,
        nach einer konfigurierten Zeit wieder aus.
        Um die Kerze manuell auszuschalten, muss sie um 180°, bzw. \textit{über Kopf} gedreht werden.
        
    \subsection{Dekorationsmodus}
        Im Dekorationsmodus wird die Kerze durch Kippen nach hinten eingeschaltet.
        Nach einer konfigurierten Zeit schaltet sie sich automatisch wieder aus.
        Um die Kerze manuell auszuschalten, muss sie um 180°, bzw. \textit{über Kopf} gedreht werden.

    \subsection{Konfigurationmodus}
        Um in den Konfigurationmodus zu wechseln, wird die Kerze nach vorne gekippt.
        Nun können alle Parameter der Kerze eingestellt werden (Siehe \ref{Konfiguration}).
        Ein Wert wird über Links- bzw. Rechtskippen gewechselt. Soll der Wert bestätigt werden,
        wird die Kerze wieder nach vorne gekippt. Die Kerze fängt an zu blinken und signalisiert
        damit, dass der Wert eingestellt wurde. Wird die Kerze wieder zurück in 
        ihre Ausgangsposition gestellt, hört sie auf zu blinken und 
        ist bereit den nächsten Wert zu konfigurieren.
        Nach dem letzten Konfigurationsschritt schaltet sich die Kerze ab
        und kann nun in einen anderen Modus versetzt werden.
        Um den Konfigurationmodus zu verlassen, müssen alle Konfigurationsschritte
        bestätigt werden.